%%%%%%%%%%%%%%%%%%%% author.tex %%%%%%%%%%%%%%%%%%%%%%%%%%%%%%%%%%%
%
% sample root file for your "contribution" to a contributed volume
%
% Use this file as a template for your own input.
%
%%%%%%%%%%%%%%%% Springer %%%%%%%%%%%%%%%%%%%%%%%%%%%%%%%%%%


% RECOMMENDED %%%%%%%%%%%%%%%%%%%%%%%%%%%%%%%%%%%%%%%%%%%%%%%%%%%
\documentclass[graybox]{svmult}

% choose options for [] as required from the list
% in the Reference Guide

\usepackage{type1cm}        % activate if the above 3 fonts are
                            % not available on your system
%
\usepackage{makeidx}         % allows index generation
\usepackage{graphicx}        % standard LaTeX graphics tool
                             % when including figure files
\usepackage{multicol}        % used for the two-column index
\usepackage[bottom]{footmisc}% places footnotes at page bottom


\usepackage{newtxtext}       % 
\usepackage{newtxmath}       % selects Times Roman as basic font

\usepackage{url} % Use the package "url.sty" to avoid problems with special characters used in your e-mail or web address

% MY ADDENDA
\usepackage{enumitem} %roman numbers
\usepackage[super]{nth} % for dates
\usepackage{verbatim} % \begin and \end {comment}



% see the list of further useful packages
% in the Reference Guide

\makeindex             % used for the subject index
                       % please use the style svind.ist with
                       % your makeindex program

%%%%%%%%%%%%%%%%%%%%%%%%%%%%%%%%%%%%%%%%%%%%%%%%%%%%%%%%%%%%%%%%%%%%%%%%%%%%%%%%%%%%%%%%%

\begin{document}

\title*{An Agent-Based Model of COVID-19 Diffusion to Plan and Evaluate Intervention Policies}
\titlerunning{An Agent-Based Model of COVID-19 Diffusion}
% Use \titlerunning{Short Title} for an abbreviated version of
% your contribution title if the original one is too long
\author{Gianpiero Pescarmona, Pietro Terna, Alberto Acquadro, Paolo Pescarmona, Giuseppe Russo, Emilio Sulis, and Stefano Terna}
 \authorrunning{G. Pescarmona, P. Terna \textit{et al.}}
% Use \authorrunning{Short Title} for an abbreviated version of
% your contribution title if the original one is too long
\institute{G. Pescarmona \at University of Torino, Italy, \email{gianpiero.pescarmona@unito.it}
\and P. Terna \at University of Torino, Italy; Fondazione Collegio Carlo Alberto, Italy \email{pietro.terna@unito.it}
\and A. Acquadro \at University of Torino, Italy \email{alberto.acquadro@unito.it}
\and P. Pescarmona \at University of Groningen, The Netherlands \email{p.p.pescarmona@rug.nl}
\and G. Russo \at Centro Einaudi, Torino, Italy \email{russo@centroeinaudi.it}
\and E. Sulis \at University of Torino, Italy; \email{emilio.sulis@unito.it}
\and S. Terna \at \url{tomorrowdata.io} \email{stefano.terna@tomorrowdata.io}
}
%
% Use the package "url.sty" to avoid
% problems with special characters
% used in your e-mail or web address
%
\maketitle

\abstract*{A model of interacting agents, following plausible behavioral rules into a world where the Covid-19 epidemic is affecting the actions of everyone. The model works with (i) infected agents categorized as symptomatic or asymptomatic and (ii) the places of contagion specified in a detailed way. The infection transmission is related to three factors: the infected person's characteristics and the susceptible one, plus those of the space in which contact occurs. The model includes the structural data of Piedmont, an Italian region, but we can easily calibrate it for other areas. The micro-based structure of the model allows factual, counterfactual, and conditional simulations to investigate both the spontaneous or controlled development of the epidemic.\newline\indent
The model is generative of complex epidemic dynamics emerging from the consequences of agents' actions and interactions, with high variability in outcomes, but frequently with a stunning realistic reproduction of the successive contagion waves in the reference region. 
There is also an inverse generative side of the model, coming from constructing a meta-agent optimizing the vaccine distribution among people groups---characterized by age, fragility, work conditions---to minimize the number of symptomatic people, using genetic algorithms.}

\abstract{A model of interacting agents, following plausible behavioral rules into a world where the Covid-19 epidemic is affecting the actions of everyone. The model works with (i) infected agents categorized as symptomatic or asymptomatic and (ii) the places of contagion specified in a detailed way. The infection transmission is related to three factors: the infected person's characteristics and the susceptible one, plus those of the space in which contact occurs. The model includes the structural data of Piedmont, an Italian region, but we can easily calibrate it for other areas. The micro-based structure of the model allows factual, counterfactual, and conditional simulations to investigate both the spontaneous or controlled development of the epidemic.\newline\indent
The model is generative of complex epidemic dynamics emerging from the consequences of agents' actions and interactions, with high variability in outcomes, but frequently with a stunning realistic reproduction of the successive contagion waves in the reference region. 
There is also an inverse generative side of the model, coming from constructing a meta-agent optimizing the vaccine distribution among people groups---characterized by age, fragility, work conditions---to minimize the number of symptomatic people, using genetic algorithms.}

%%%%%%%%%%%%%%%%%%%%%%%%%%%%%%%%%%%%%%%%%%%%%%%
%%%%%%%%%%%%%%%%%%%%%%%%%%%%%%%%%%%%%%%%%%%%%%%
\section{A quick introduction to epidemic modeling}
\label{intro}

The starting point from which we built our model is that of S.I.R compartmental models with Susceptible, Infected, and Recovered people. This approach allows looking at the epidemic dynamics, but on a macro scale. While the Covid-19 epidemic was spreading, there have been several attempts to introduce more realistic compartmental subdivisions. A relevant example in this direction is that of \cite{Scala:2020aa}. The research also been followed other work lines, as in \cite{doi:10.1142/S0218202520500323}, where a multiscale framework accounts for the interaction of different spatial levels, from the small scale of the virus itself and cells, to the large scale of individuals and further up to the collective behavior of populations.

Following \cite{rahmandad2008heterogeneity}, we know that the analysis based on the assumption of  heterogeneity strongly differs from S.I.R. compartmental structures modeled by differential equations. Their work ponders when it is better to use agent-based models and when it would be better to use differential equation models. Differential equation models assume homogeneity and perfect mixing within compartments, while agent-based models can capture heterogeneity in agent attributes and the structure of their interactions. We follow the second approach.

Finally, we subscribe the call of \cite{squazzoni2020} to <<cover the full behavioural and social complexity of societies under pandemic crisis>> and we move in that directionin our work reported here. 

%%%%%%%%%%%%%%%%%%%%%%%%%%%%%%%%%%%%%%%%%%%%%%%
\subsection{Why model? Why agents? Why another model?}
\label{why}

Why another model, and most of all, with \cite{epstein2008model}, why model? With the author, the reply is: 
\begin{quote}
The choice (\ldots) is not whether to build models; it's whether to build explicit ones. In explicit models, assumptions are laid out in detail, so we can study exactly what they entail. On these assumptions, this sort of thing happens. When you alter the assumptions that is what happens. By writing explicit models, you let others replicate your results.
\end{quote}

And, strongly:
\begin{quote} 
I am always amused when these same people challenge me with the question,``Can you validate your model'' The appropriate retort, of course, is,``Can you validate yours?'' At least I can write mine down so that it can, in principle, be calibrated to data, if that is what you mean by ``validate'' a term I assiduously avoid.
\end{quote}

To reply to ``why agents?'', with \cite{axtell2000agents} we define in short what an agent-based model is:
\begin{quote} 
An agent-based model consists of individual agents, commonly implemented in software as objects. Agent objects have states and rules of behavior. Running such a model simply amounts to instantiating an agent population, letting the agents interact, and monitoring what happens. That is, executing the model---spinning it forward in time---is all that is necessary in order to ``solve'' it.
\end{quote}

More in detail:
\begin{quote} 
There are, ostensibly, several advantages of agent-based computational modeling over conventional mathematical theorizing. First, as described above, it is easy to limit agent rationality in agent-based computational models. Second, even if one wishes to use completely rational agents, it is a trivial matter to make agents heterogeneous in agent-based models. One simply instantiates a population having some distribution of initial states, e.g., preferences. That is, there is no need to appeal to representative agents. Third, since the model is ``solved'' merely by executing it, there results an entire dynamical history of the process under study. That is, one need not focus exclusively on the equilibria, should they exist, for the dynamics are an inescapable part of running the agent model. Finally, in most social processes either physical space or social networks matter. These are difficult to account for mathematically except in highly stylized ways. However, in agent-based models it is usually quite easy to have the agent interactions mediated by space or networks or both.
\end{quote}

And now, "why another?" As a commitment to our creativity, using our knowledge to understand what is happening. Indeed, with arbitrariness: it is up to others and time to judge. Modeling the Covid-19 pandemic requires a scenario and the actors.
As in every play, the author defines the roles of the actors and the environment. The characters are not real, they are pre-built by the author, and they act according to their peculiar constraints. If the play is successful, they will play for a long time, even centuries. If not, we will rapidly forget them. Shakespeare Hamlet is still playing after centuries, even if he is entirely imaginary.

The same holds for our simulations: we are the authors, we arbitrarily define the characters, we force them to act again and again in different scenarios. In both plays and simulations, we compress the time: whole life to 2 or 3 hours on the stage. In a few seconds, we run the Covid-19 pandemic spread in a given regional area.

%%%%%%%%%%%%%%%%%%%%%%%%%%%%%%%%%%%%%%%%%%%%%%%
\subsection{Our model}
\label{ourModel}

With our model, we move from a macro compartmental vision to a meso and microanalysis capability. Its main characteristics are:

\begin{itemize}

\item
scalability: we take in account the interactions between virus and molecules inside the host, the interactions between individuals in more or less restricted contexts, the movement between different environments (home, school, workplace, open spaces, shops, \ldots);\footnote{In a second version, we will add transportations and long trips between regions/countries; discotheques; other social aggregation, as football events.}

in detail, the scales are: 

\begin{itemize}
\setlength\itemsep{0.3em}
\item
	\emph{micro}, with the internal biochemical mechanism involved in reacting to the virus, as in \cite{Silvagno_2020}, from where we derive the critical importance assigned to an individual attribute of intrinsic susceptibility related to the age and previous morbidity episodes; the model indeed incorporates the medical insights and consistent perspectives of one of its co-authors, former full professor of clinical biochemistry, signing also the quoted article; a comment on Lancet \cite{horton2020offline} consistently signals the syndemic character of the current event: <<Two categories of disease are interacting within specific populations---infection with severe acute respiratory syndrome coronavirus 2 (SARS-CoV-2) and an array of non-communicable diseases (NCDs)>>;
\item
	\emph{meso}, with the open and closed contexts where the agents behave, as reported above;
\item	
	\emph{macro}, with the emergent effects of the actions of the agents;
	
\end{itemize}

\item
granularity: at any level, the interactions are partially random and therefore the final results will always reflect the sum of the randomness at the different levels; changing the constraints at different levels and running multiple simulations should allow the identification of the most critical points, where to focus the intervention.

\end{itemize}

Summing up, S.I.s.a.R. \cite{SIsaR} is an agent-based model designed to reproduce the diffusion of the COVID-19 using agent-based modeling in NetLogo \cite{NetLogo}. We have Susceptible, Infected, symptomatic, asymptomatic, and Recovered people: hence the name S.I.s.a.R.. The scheme comes from S.I.R. models, herewith (i) infected agents categorized as symptomatic and asymptomatic and (ii) the places of contagions specified in a detailed way, thanks to agent-based modeling capabilities. The model includes Piedmont's structural data, an Italian region, but we can quite easily calibrate it for other areas. It can reproduce the events following a realistic calendar (e.g., national or local government decisions, as in Appendix \ref{appCalendar}), via its script interpreter.\footnote{\label{modOnLine}The model is online at \url{https://terna.to.it/simul/SIsaR.html}, from where it is also possible to run the code without installation. Corresponding author: Pietro Terna: \url{mailto:pietro.terna@unito.it}. Looking at the \emph{info} sheet of the model, you have more than 20 pages of Supporting Information about both the structure and the calibration of the model.}

We place two initial infected individuals in a population of 4350 individuals, on a scale of 1:1000 with Piedmont.\footnote{They appear as black segments in the sequences of Appendix \ref{appContagion}.} The size of the initial infected group is out of scale: it is the smallest number, ensuring the epidemic's activation in a substantial number of cases. Initial infected people bypass the incubation period. For implausibility reasons, we never choose initial infected people among persons in nursing homes or hospitals. The presence of agents in close space---such as classrooms, factories, homes, hospitals, nursing homes---is made with realistic numbers, not to be read in scale: e.g., a classroom contains 25 students, a home two persons, etc.; the movements occur in different parts of the daily life, as in \cite{ghorbani2020assocc}.

We can set: 
\begin{itemize}
\setlength\itemsep{0.3em}
\item min and max duration of the individual infection;

\item the length of the incubation interval;

\item the critical distance, as the radius of a circle affecting people which are in it, with a given probability;

\item the correction of that probability, due to the personal characteristics of both active and the passive agents; passive agents, as receivers, can be robust, regular, fragile, and extra fragile.

\end{itemize} 

We have two main types of contagion: (a) within a radius, for people moving around, also if only temporary present in a house/factory/nursing home/hospital (in schools we only have students and teachers); (b) in a given space (room or apartment) for people resident in their home or in a hospital or in a nursing home or being in school or in a working environment.

People in hospitals and nursing homes can be infected in two ways: (a) and (b). Instead, while people are at school, they can only receive the disease from people in the same classroom, where only teachers and students are present, so this is a third infection mechanism (c).

One should remark that workplaces are open to all persons, as clients, vendors, suppliers, external workers can go there. In contrast, schools are mainly reserved for students and school operators and are less affected by contact with other types of agents.

All agents have their home, inside a city, or a town. The agents also have a regular place (RP) where they act and interact, moving around. These positions can be interpreted as free time elective places. When we activate the school, students and teachers have both RPs and the schools; healthcare operators have both RPs and hospitals or nursing homes; finally, workers have both RPs and working places. In each day (or tick of the model), we simulated realistic sequences of actions.




%%%%%%%%%%%%%%%%%%%%%%%%%%%%%%%%%%%%%%%%%%%%%%%
%%%%%%%%%%%%%%%%%%%%%%%%%%%%%%%%%%%%%%%%%%%%%%%
\section{How S.I.s.a.R. works}
\label{howWorks}

%%%%%%%%%%%%%%%%%%%%%%%%%%%%%%%%%%%%%%%%%%%%%%%
\subsection{A day of the simulation}\label{aDay}

Figure \ref{outline} describes what happens during every day in our simulated world.

\begin{figure}[t]
\center
\includegraphics[width=1\textwidth,,angle=0]{SIsaR_outline.png} %}
\caption{A day in the simulation, with $N$ repetition where $N$ is the duration of a specific outbreak; look at Section \ref{cond} for the conditional actions,  Section \ref{par} for the parameters, and Section \ref{inter} for details on the interactions}
\label{outline}
\end{figure}


%%%%%%%%%%%%%%%%%%%%%%%%%%%%%%%%%%%%%%
\subsection{Conditional actions}
\label{cond}

\begin{description}

\item [condition I] Symptomatic persons are at home or in a hospital or a nursing home and do not move. 

\item [condition II] People not constrained by \emph{condition I} can move if (basic rule) no limitations/lockdown \emph{OR} one of the following situations:

\begin{enumerate}
\item hospital healthcare operators, nursing home healthcare operators;

\item all people as \verb|%PeopleAnyTypeNotSymptomaticLeavingHome| 
(\ref{par}, \ref{p3});

\item regular people as \verb|%PeopleNotFragileNotSymptomaticLeavingHome|
(\ref{par}, \ref{p4};

\item workers following \verb|%openFactoriesWhenLimitationsOn|
(\ref{par}, \ref{p5});

\item teachers if \verb|activateSchool| is $true$
(\ref{par}, \ref{p6});

\item students if \verb|activateSchool| is $true$ with \%students $>0$
(\ref{par}, \ref{p7}).

\end{enumerate}

\end{description}


%%%%%%%%%%%%%%%%%%%%%%%%%%%%%%%%%%%%%%
\subsection{Parameters}\label{par}

In round brackets we report the short names if used in program scripts. About the days, please interpolate the dates of Table \ref{dates}.

\begin{enumerate}[label=\roman*]

\item \label{p1} \verb|probabilityOfGettingInfection| (\verb|prob|): 0.05 (starting phase); 0.02 at day 49 (adoption of non-pharmaceutical measures); 0.035 at day 149 (some relaxation in compliance); 0.02 at day 266 (again, compliance to rules);  

\item \label{p2} \verb|intrinsicSusceptibility| based on \verb!intrinsicSusceptibilityFactor! set to 5 in Eq. \ref{intrinsic}
\begin{equation}
intrinsicSusceptibility = intrinsicSusceptibilityFactor^{groupFragility}
\label{intrinsic}
\end{equation}
with $groupFragility$ exponent set to:

\begin{description}
\item [1] for extra-fragile persons,
\item [0] for fragile persons,
\item [-1] for regular persons,
\item [-2] young people from 0 to 24 years old;
\end{description}

\item \label{p3} \verb|%PeopleAnyTypeNotSymptomaticLeavingHome| (\verb|%PeopleAny|)
determines, in a probabilistic way, the number of all people going around in case of limitations/lockdown; 
the limitation operates only if the lockdown is on (see above); in use
at (day) 20, 90; at 28, 80; at 31, 0; at 106, 80; at 110, 95; at112, 85; at 117, 95; at 121, 90; at 259, 90; at 266, 80; at 277, 50;
at 302, 70; at 320, 90; at 325, 50; at 329, 80; at 332, 50; at 336, 80; at 337, 50; at 339, 80; 

\item \label{p4} \verb|%PeopleNotFragileNotSymptomaticLeavingHome| (\verb|%PeopleNot|)
determines, in a probabilistic way, the number of regular people going around in case of limitations/lockdown;
the limitation operates only if the lockdown is on (into our simulated world, from day 20); \verb|%PeopleNot| values: 
at (day) 31, 80; at 35, 70; at 36, 65; at 38, 15; at 42, 25; at 84, 30; at 106, 0; at 302, 90; at 325, 50; at 332, 50; 
at 337, 50; at 339, 100; at 349, 90;

the parameters \ref{p3} and \ref{p4} in some phase change very frequently, reproducing into the model the uncertainty of the decisions that was happening  in the real world in the same periods; 

NB, the parameters \ref{p3} and \ref{p4} produce independent effects, as in the following examples: (a) the activation of \emph{\%PeopleAny at 31, 0} and, simultaneously, of \emph{\%PeopleNot at 31, 80}, means that people had to stay home on that day, but people specifically not fragile could go out in 80\% of the cases; (b) \emph{\%PeopleAny at 339, 80} and, simultaneously, \emph{\%PeopleNot at 339, 100} means that fragile and not fragile persons cannot always go around, but only in the 80\% of the cases, instead considering uniquely non-fragile persons they are free to go out; the construction is an attempt to reproduce a fuzzy situation;

in future versions of the model, we will define the quotas more straightforwardly:
\begin{itemize}
\item \verb|%FragilePeopleNotSymptomaticLeavingHome|;
\item \verb|%NitFragilePeopleNotSymptomaticLeavingHome|;
\end{itemize}

\item \label{p5} \verb|%openFactoriesWhenLimitationsOn| (\verb|%Fac|) 
determines, in a probabilistic way, what factories (small and large industries, commercial surfaces, private and government offices)
are open when limitations are on; if the factory of a worker is open, the subject can go to work, avoiding restrictions (but uniquely in the first step of activity of each day); \verb|%Fac| 
is in use at (day) 38, value4 0; at 49, 20; at 84, 70; at 106, 100; at 266, 90; at 277, 70; at 302, 80; at 320, 90; at 325, 30;
at 329, 90; at 332, 30; 336, 90; at 337, 30; at 339, 100;

\item \label{p6} \verb|stopFragileWorkers|  (\verb|sFW|); if set to 1, fragile workers (i.e., people fragile due to prior illnesses) can move out of their homes following the \ref{p3} and \ref{p4} parameters, but can go to work in no case; the regular case is that the workers (fragile or regular) can go to their factory (if open) also when limitations are on; in one of the experiments we used \verb|sFW| with set to 1 (on) at day 245  and to 0 (off) at day 275;

alternatively, we also have the \verb|fragileWorkersAtHome| parameter; if on (set to 1) the total of the workers is unchanged, but the workers are all regular; we can activate this counterfactual operation uniquely at the beginning of the simulation;

\item \label{p7} when \verb|activateSchools| (\verb|aSch|) is on (set to 1) teachers and students go to school avoiding restrictions (but uniquely in the first step of activity of each day);  \verb|\%Students| limits to its value the quota of the students moving to school; the residual part is following the lessons from home; we used \verb|aSch| at (day) 1, on; at 17, off; at 225, on; at 325, off; at 339, on; we used \verb|1%st| 
at (day) 0, 100; at 277, 50; at 339, 50; at 350, 50 (repeated values are not relevant for the model, but for the memory of the programmer-author);

\item \label{p8} \verb|radiusOfInfection| (\verb|radius|) with value 0.2; the effect of the contagion---outside enclosed spaces, or there, but for temporary presences---is possible within that distance; in the model, space is missing of a scale, but forcing the area to be in the scale of a region as Piedmont, 0.2 is equivalent to 20 meters; we have to better calibrate this measure, with movements and probabilities; this is a key step in future developments of the model.

\end{enumerate}

%%%%%%%%%%%%%%%%%%%%%%%%%%%%%%%%%%%%%%
\subsection{Agents' interaction}
\label{inter}

We underline that our tool is not based on microsimulation sequences, calculating contagions agent by agent, following their characteristics and ex-ante probabilities. It implements a true agent-based simulation, with the agents acting and, most of all, interacting, thus generating contagions.

Each run creates a population following expected characteristics but with small random specifications to assure heterogeneity in agents. The daily choices of the agents are partially random to reproduce real-life variability.

Contagions arise from agents' interactions, in four situations, as specified in Fig. \ref{outline}:

\begin{enumerate}[label=\Alph*]

\item - in houses (at night), hospitals, nursing homes;

\item - in schools, workplaces in general, among people stable there;

\item - in the same places (excluding schools) by people temporary there and in open spaces;

\item - interactions mainly in open spaces.

\end{enumerate}

At \url{https://terna.to.it/simul/contagionSequences.pdf}, in Section ``2 The visualization of the sequences of contagions in simulated epidemics'' and in ``A Appendix: Analyzing examples of contagion sequences'' we have the visualizations of the effects of those sequences of interactions-contagions.


%%%%%%%%%%%%%%%%%%%%%%%%%%%%%%%%%%%%%%
\subsection{Model calendar}\label{calend}

Day 1 is fixed at Feb \nth{4}, 2020. The calendar scan is that of the Table \ref{dates}.

\begin{table}[t]
\begin{footnotesize}
\begin{tabular}{rr}
 Day & Date  \\
 25 & 28- 2-2020 \\
 50 & 24- 3-2020 \\
 75 & 18- 4-2020 \\
100 & 13- 5-2020 \\
125 &  7- 6-2020 \\
150 &  2- 7-2020 \\
175 & 27- 7-2020 \\
200 & 21- 8-2020 \\
225 & 15- 9-2020 \\
250 & 10-10-2020 \\
275 &  4-11-2020 \\
300 & 29-11-2020 \\
325 & 24-12-2020 \\
350 & 18- 1-2021 \\
375 & 12- 2-2021 \\
400 &  9- 3-2021 \\
425 &  3- 4-2021 \\
450 & 28- 4-2021 \\
475 & 23- 5-2021 \\
500 & 17- 6-2021 \\
525 & 12- 7-2021 \\
550 &  6- 8-2021 \\
575 & 31- 8-2021 \\
600 & 25- 9-2021 \\
625 & 20-10-2021 \\
650 & 14-11-2021 \\
675 &  9-12-2021 \\
\end{tabular}
\end{footnotesize}
\caption{The days of the simulation and their position in the calendar}
\label{dates}
\end{table}


%%%%%%%%%%%%%%%%%%%%%%%%%%%%%%%%%%%%%%%%%%%%%%%
%%%%%%%%%%%%%%%%%%%%%%%%%%%%%%%%%%%%%%%%%%%%%%%
\section{The visualization of the sequences of contagions in simulated epidemics}
\label{visualization}

%%%%%%%%%%%%%%%%%%%%%%%%%%%%%%%%%%%%%%%%%%%%%%%
\subsection{Contagion sequences}
\label{sequences}

How to understand what is happening in our simulated epidemics at a micro-scale? The key idea is to analyze the sequences of contagions by representing each infecting agent as a horizontal segment with a vertical link to another agent, receiving the infection. We render it via another segment at an upper layer. With colors, line thickness, and styles, we represent multiple data. We have time on the $x$ axis and the progressive ordinal number of the infected agents in the $y$ axis.

Read about the detail of visualization technique in Appendix \ref{appHowItWorks} and in the example of Section \ref{howToAnalyze}. At \url{https://github.com/terna/contagionSequence} we have the sequence analyzer. From there, you can also run the program automatically, thanks to \url{https://mybinder.org}.

Looking at the different sequences, one feels as \emph{The Sorcerer's Apprentice} of the Goethe 1797 poem. How to proceed?


%%%%%%%%%%%%%%%%%%%%%%%%%%%%%%%%%%%%%%%%%%%%%%%
\subsection{A few sequences suggesting a policy via counterfactual limitations}
\label{seqSuggPol}

We report several sequences in Appendix \ref{appContagion}, considering them mainly as examples to comment, examining the effects of nursing homes, workplaces, hospitals, homes, luckily close to never schools. Among those cases, we highlight the inspiring sequence of Section \ref{c8} topics.

In Fig. \ref{fourSequences} we can look both at the places where contagions occur and at the dynamics emerging with different levels of intervention. Using the article's pdf version as a file, the reader can enlarge the four pictures (and any figure in the appendices). The reference to specific days is related to the calendar of Appendix \ref{appCalendar}. Here, in the fourth case (bottom right of Fig. \ref{fourSequences}, we introduce the stop to fragile agents of any type at Feb \nth{15}; the decision would have been plausible, considering that the situation of danger probably was known before that date. To be more realistic, the analysis that deepens that situation in Appendix \ref{app1000run} and so in Table \ref{keyResultsT} uses the day Feb \nth{20} as a turning point.

\begin{figure}[t]
\center
\includegraphics[scale=0.16]{withShort1.png}~~~\includegraphics[scale=0.16]{withShort1A.png} 
%\hspace{0.2cm}
\center
\includegraphics[scale=0.16]{withShort1A200.png}~~~\includegraphics[scale=0.16]{withShort1B.png} \\
\caption{(\emph{top left}) an epidemic with containment measures, showing a highly significant effect of workplaces (brown);
 (\emph{top right}) the effects of stopping fragile workers at day 20, with a positive result, but home contagions (cyan) keep alive the pandemic, exploding again in workplaces (brown); (\emph{bottom left}) the same analyzing the first 200 infections with evidence of the event around day 110 with the new phase due to a unique asymptomatic worker, and (\emph{bottom right}) stopping fragile workers and any case of fragility at day 15, also isolating nursing homes} 
\label{fourSequences}
\end{figure}

\begin{comment}
withShort1.png         %using SIsaR 9.4.2 as is (basic control) with 123456_22314, file ex123456_22313.csv;
withShort1A.png      %using SIsaR 9.4.2 as is (basic control + 20 sFW 1,i.e., stop Fragile Workers) with 123456_22314, file ex123456_22314A.csv;
withShort1A200.png %using SIsaR 9.4.2 as is (basic control + 20 sFW 1,i.e., stop Fragile Workers) with 123456_22314, file ex123456_22314A.csv;
withShort1B.png       %using SIsaR 9.4.2 as is (basic control + $$$) with 123456_22314, file ex123456_22314B.csv;
\end{comment}

The four pictures, related to epidemics starting precisely in the same way, represent an evolving narrative, that:

\begin{enumerate}

\item starts from the observation of an epidemic in which workplaces have an evident role in sustaining the spreading of the virus, despite the adoption of the non-pharmaceutical containment measures adopted locally and at the national level;

\item adopts a counterfactual limitation holding back fragile workers from factories (any workplace), with some initial success, but with a \emph{bridge} to a phase 2;

\item\label{micromicro} deepens the situation of the specific agent operating as a bridge, a regular (non-fragile) worker infected at work by another regular worker infected at home by a fragile agent;

\item introduces a more substantial  control, anticipating at Feb \nth{15} the limitation to fragile workers and stopping the mobility of all fragile people from Feb \nth{20} with evident positive effect, having the whole epidemic very few contagions and lasting a limited number of days.

\end{enumerate}

In our model, the fragile workers are those 55 years old or more; in this scheme, if they cannot work remotely from home, they are supposed to obtain regular sick pay (see Sections \ref{CBanalysis} and \ref{c8} for considerations).

This kind of analysis is a source of suggestions for interventions, also if we cannot validate them only with micro studies, as in bullet point \ref{micromicro} above. 

Summarizing: 

\begin{itemize}
\setlength{\itemsep}{0pt}
\item we confirm the interest of the knowledge that we can extract from contagion sequences; 
\item we suggest, as an integrative example, the simulation of Fig. \ref{6b}  in Appendix \ref{appContagion}, showing many cases of fragile workers diffusing the infection;
\item finally, we will use a more systematic data exploration, in Section \ref{app1000run}, as summarized in Section \ref{keyResultsS}.

\end{itemize}


\bibliography{./bibliografiaGenerale}
\bibliographystyle{spphys}

\end{document}
